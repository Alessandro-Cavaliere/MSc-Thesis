\chapter*{Acknowledgments}
\addcontentsline{toc}{chapter}{Acknowledgments}

\begingroup

\setlength{\parindent}{0pt}     
\setlength{\parskip}{6pt plus2pt}
\newcommand{\separator}{
  \par\medskip\hrule\medskip
}

Desidero innanzitutto esprimere la mia più profonda gratitudine al \textbf{Professore Esposito}, guida lucida e costante durante quest’ultimo anno di magistrale. La ringrazio per avermi orientato nella scelta delle diverse mete Erasmus, per i preziosi suggerimenti offerti con generosità e per i numerosi confronti in cui, pur partendo da prospettive talvolta divergenti, ha saputo arricchire la mia formazione accademica e personale. Sono certo che, anche oltre i confini universitari, continueremo a collaborare là dove si presenteranno nuove opportunità di progetti dal valore accademico condiviso.

\separator

I would also like to extend my heartfelt thanks to \textbf{Professors Nuno Laranjeiro} and \textbf{Naghmeh Ivaki}. From the very beginning of my Erasmus experience, you made me feel at home and were consistently understanding during the initial orientation at the university and whenever any issues arose. Your feedback, methodological rigor, and warm enthusiasm proved invaluable throughout this project. Working under your guidance broadened my horizons and pushed me to attain a level of academic maturity that I will carry forward forever in my future endeavors. I will always remember the beautiful city of Coimbra (that I will definitely come back to visit again), and I know that even your small contribution helped make my time there a wonderful and unforgettable experience.

\separator

Un pensiero ovviamente è riservato ai \textbf{miei genitori} e a \textbf{mio fratello}. Mi avete sostenuto in ogni decisione, avete condiviso le mie preoccupazioni e gioito dei miei traguardi. La vostra presenza discreta ma costante ha costituito un porto sicuro nei momenti più complessi di questo percorso magistrale. Sono sicuro di avervi reso orgogliosi di me. Vi voglio un mondo di bene... e sbaglio a non dirvelo abbastanza.

\separator

Un grazie speciale va alla mia ragazza, \textbf{Giada}. Grazie per la tua pazienza infinita nel sopportare il mio carattere a volte fin troppo testardo, per la motivazione dolce ma determinata che sai trasmettermi e per la fiducia incrollabile che riponi nelle mie capacità. I tuoi incoraggiamenti, anche nei momenti in cui affrontavi le tue difficoltà, hanno reso più leggere le mie giornate di studio e di lavoro, ricordandomi sempre per chi e per cosa valeva davvero la pena impegnarsi. Ti amo da morire, sei la mia vita.

\separator

Un piccolo pensiero va al mio compagno di studi per eccellenza, \textbf{Carmine}. Senza di te questo percorso sarebbe stato un viaggio in solitaria: abbiamo condiviso ogni tappa, dalle scelte lavorative a quelle accademiche, affrontando insieme la maggior parte delle sfide che si sono presentate, celebrando la gran parte dei traguardi insieme. Hai reso questo cammino più leggero, divertente e decisamente meno stressante. Grazie per la tua costante presenza, per il sostegno, per esserci stato e per continuare ad esserci, non solo come collega di studi, ma soprattutto come amico. Ti voglio bene.

\separator

Ringrazio tutti i miei \textbf{amici} che, pur non avendo condiviso direttamente il mio percorso di studi, sono sempre stati pronti per una pizza improvvisata, una passeggiata in centro o una serata spensierata. La vostra presenza concreta e la capacità di farmi staccare la spina nei momenti di maggiore pressione quando ne sentivo il bisogno, sono stati un grande aiuto. Grazie per aver rispettato i miei ritmi e le mie assenze, e soprattutto per non aver mai vacillato, neppure dopo i lunghi periodi in cui non ci vedevamo. Sono ormai tanti anni che vi conosco, siamo praticamente cresciuti insieme. Qualunque strada imboccheremo e qualsiasi grande impegno che ci toglierà tempo, spero che sapremo ritrovarci sempre, per festeggiare insieme altri traguardi come questo, miei e vostri.

\separator

Per ultimo, ma non per importanza, vorrei ringraziare il \textbf{me stesso} del passato. A quel me che, nonostante un anno di lavoro a tempo pieno, i tanti esami da sostenere nel frattempo, i corsi extracurricolari che ho sostenuto e che mi hanno dato tante soddisfazioni e al mio coraggio di uscire da quella comfort zone; ho saputo stringere i denti, organizzare il tempo con disciplina e affrontare ogni ostacolo e nuova esperienza con dedizione e tanta tenacia. Ho versato tanto (forse troppo) sudore, sacrificato serate e weekend, ma non me ne pento. Ho imparato tanto, sono migliorato molto e ho raggiunto risultati che parlano di impegno costante e soprattutto di una passione autentica. Questo traguardo è solo una piccola punta di iceberg della prova tangibile che la perseveranza, alimentata da obiettivi chiari e da una forte volontà, può trasformare le difficoltà e i forti impegni, in opportunità di crescita e che "\textit{If you never try, you'll never know just what you are worth}" cit.  Sono orgoglioso di questo cammino, dei vari traguardi che sono riuscito a compiere e dello slancio con cui guardo ai prossimi capitoli della mia vita.

\begin{flushright}
\textit{Ad maiora.}
\end{flushright}
\endgroup