\begingroup
\selectlanguage{english}
\phantomsection
\addcontentsline{toc}{chapter}{Abstract}
\begin{abstract}
\thispagestyle{plain}
Vulnerability disclosure is a very complex process in the field of blockchain and cybersecurity. While software manufacturers want to protect their systems and preserve their reputations, vulnerability researchers want to prove that security problems exist. However, communication between researchers and companies can be difficult, as not all companies react promptly or offer adequate rewards for discoveries, creating frustration among researchers. This friction can lead to tensions in the disclosure process, with some companies choosing to ignore security reports or even fail to acknowledge potential risks.
In the context of decentralized systems such as DeFi, these problems are even more severe. The open and immutable nature of smart contracts and the absence of centralized control further complicates the vulnerability management process. Bug-bounty programmes, for instance, are intended to be a cornerstone of responsible disclosure, providing researchers with a legal, compensated path to report flaws and giving project teams a structured workflow for triage and patching; yet many of these programmes are poorly managed, with some companies failing to honour rewards or offering token compensation, which discourages researchers from submitting critical vulnerabilities and leaves serious threats unaddressed. The situation is exacerbated by an influx of low-quality, AI-generated reports or bot spam that overwhelm triage teams and obscure legitimate findings.

A zero-knowledge approach can restore trust: by equipping researchers with a tool that produces a succinct proof showing that a specific transaction triggers an unintended state change, without revealing the exploit’s technical details, both parties gain. Researchers can demonstrate exploitability and claim a bounty without risking premature disclosure, while project owners receive irrefutable evidence of risk and can release rewards promptly, confident that sensitive information remains hidden until a fix is deployed.

This thesis presents \textit{zkpoex}, a ready-to-use Rust framework that generates zero-knowledge \textit{proofs of exploit}. With a single command, the tool replays a confirmed exploit transaction, the exact calldata and value that cause the bug inside RISC Zero’s zkVM, checks user-defined safety conditions, and produces a compact receipt that can be verified on-chain. In practice, a researcher can prove that their malicious transaction really breaks the contract and immediately claim the bounty, all without disclosing the underlying attack code; Meanwhile, the project gains cryptographic proof of the bug before any details leak, turning bounties into trustless, automatic agreements and raising the bar for collaborative vulnerability management in web3.
\end{abstract}
\endgroup


