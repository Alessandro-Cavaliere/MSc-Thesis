\usepackage[utf8]{inputenc}
\usepackage[big]{titlesec}
\usepackage[italian,english]{babel}
\usepackage{graphicx}
\usepackage{cite}
\usepackage{amssymb}
\usepackage{amsmath}
\usepackage[table,xcdraw]{xcolor}
\usepackage[italian]{minitoc}% per gli accenti
\usepackage{fancybox}
\usepackage{fancyhdr}
\usepackage{verbatim}
\usepackage{url}
\usepackage{color}
\usepackage{xcolor}
\usepackage{listings}
\usepackage{makeidx}
\usepackage{comment}
\usepackage[a4paper, total={5in,8in}]{geometry}
\usepackage{etoolbox}
\usepackage{tabularx}  

\usepackage{epigraph}         
\setlength{\epigraphwidth}{0.80\textwidth}  
\setlength{\epigraphrule}{0pt}          
\renewcommand{\textflush}{flushright}      
\renewcommand{\epigraphsize}{\Large\itshape}
\renewcommand{\sourceflush}{flushright}  


\renewcommand{\arraystretch}{1.2} 
\DeclareTextFontCommand{\textcourier}{\fontfamily{phv}\selectfont}

\renewcommand\lstlistlistingname{Code Listings}

\newcommand\myemptypage{
    \null
    \thispagestyle{empty}
    \addtocounter{page}{-1}
    \newpage
    }
    
    \newcommand\emptypage{
    \null
    \addtocounter{page}{-1}
    \newpage
    }

\makeatletter
\newcommand*\bigcdot{\mathpalette\bigcdot@{.5}}
\newcommand*\bigcdot@[2]{\mathbin{\vcenter{\hbox{\scalebox{#2}{$\m@th#1\bullet$}}}}}
\makeatother
%%%%Url settings%%%%%%%
\usepackage[breaklinks=true]{hyperref} 
\def\UrlBreaks{\do\/\do-\do\_}

%%%%%%%%%%%%%%%%%

%%%%Footnote settings%%%%
\usepackage{chngcntr}
\counterwithout{footnote}{chapter}
\renewcommand{\thefootnote}{\textbf{\arabic{footnote}}}
\usepackage[flushmargin]{footmisc}
\addtolength{\footnotesep}{1mm} % change to 1mm
%%%%%%%%%%%%%%%%%

\renewcommand{\citeform}[1]{\textbf{#1}}



\pagestyle{plain}
\fancyhf{}

%Definizione var footnotemark per poter colorare i numeri riferiti ai footnote
\makeatletter
\def\@footnotecolor{red}
\define@key{Hyp}{footnotecolor}{%
 \HyColor@HyperrefColor{#1}\@footnotecolor%
}
\patchcmd{\@footnotemark}{\hyper@linkstart{link}}{\hyper@linkstart{footnote}}{}{}
\makeatother


%Settaggio colori e stili per link,footnote e url
\hypersetup{
colorlinks=true,
linkcolor=red,     
urlcolor=blue,
citecolor=[RGB]{0,0,255},
footnotecolor=red,
}
\usepackage{enumitem}

%%%%%%%%%%%%%%%%%%%%%%%%%
%Settaggio colori e funzioni per 
\usepackage{listings} %code highlighter
\usepackage{upquote}
\usepackage{color}
\definecolor{editorGray}{rgb}{0.95, 0.95, 0.95}
\definecolor{editorOcher}{rgb}{1, 0.5, 0} % #FF7F00 -> rgb(239, 169, 0)    
\definecolor{editorGreen}{rgb}{0, 0.5, 0} % #007C00 -> rgb(0, 124, 0)
\definecolor{mygreen}{rgb}{0,0.6,0}
\definecolor{mygray}{rgb}{0.5,0.5,0.5}
\definecolor{mymauve}{rgb}{0.58,0,0.82}
\definecolor{yellowFunction}{RGB}{247,173,78}
\definecolor{greenComment}{RGB}{89,127,70}
\definecolor{orangeWords}{RGB}{184,94,45}
\definecolor{purpleWords}{RGB}{138,110,158}
\definecolor{greyWords}{RGB}{80,80,83}


%% ----------------------------------------
%% Custom listings: JSX (already present),
%% plus Rust and TOML for Chapter 3
%% ----------------------------------------

%% ----------------------------------------------------------
%% Vivid listings for Rust, Solidity and TOML
%% ----------------------------------------------------------
\definecolor{kwRust}   {rgb}{0.02,0.28,0.82}   % strong cobalt
\definecolor{ndRust}   {rgb}{0.90,0.30,0.07}   % bright vermilion
\definecolor{kwSol}    {rgb}{0.10,0.55,0.10}   % vivid green
\definecolor{ndSol}    {rgb}{0.75,0.34,0.96}   % purple
\definecolor{kwToml}   {rgb}{0.82,0.42,0.04}   % orange
\definecolor{commentFG}{rgb}{0.38,0.38,0.38}   % mid-grey
\definecolor{stringFG} {rgb}{0.02,0.56,0.45}   % teal

%% ---------------------------  Rust  ---------------------------
\lstdefinelanguage{rust}{
  sensitive=true,
  morekeywords={
    as,async,await,break,const,continue,crate,else,enum,extern,false,fn,for,
    if,impl,in,let,loop,match,mod,move,mut,pub,ref,return,self,Self,static,
    struct,super,trait,true,type,unsafe,use,where,while,
    Result,Option,Some,None,Ok,Err,&str,,Box,decode,unwrap,transact_call,new,assert,matches,derive,Serialize,Deserialize,tokio,test,run\_evm
  },
  % highlight attribute identifiers rather than the literal "#["
  emph={},
  emphstyle=\color{ndRust}\bfseries,
  keywordstyle   = \color{kwRust}\bfseries,
  ndkeywords     ={u8,u16,u32,u64,usize,Vec,hex,i8,i16,i32,i64,isize,
                   f32,f64,bool,char,String,U256,B256,H160,H256},
  ndkeywordstyle = \color{ndRust}\bfseries,
  identifierstyle= \color{black},
  comment=[l]{//},
  morecomment=[s]{/*}{*/},
  commentstyle   = \color{commentFG}\ttfamily,
  stringstyle    = \color{stringFG}\ttfamily,
  morestring=[b]",
}


%% -------------------------  Solidity  -------------------------
\lstdefinelanguage{solidity}{
  sensitive=true,
  morekeywords={
    pragma,solidity,contract,interface,library,import,using,is,struct,enum,
    event,modifier,function,constructor,fallback,receive,return,returns,
    mapping,storage,memory,calldata,if,else,for,while,do,break,continue,throw,
    try,catch,emit,revert,assembly,let,switch,case,default,public,external,
    internal,private,view,pure,payable,const,override,virtual,immutable,
    anonymous,indexed
  },
  keywordstyle   = \color{kwSol}\bfseries,
  ndkeywords={
    address,bool,string,bytes,byte,bytes1,bytes2,bytes3,bytes4,bytes5,
    bytes6,bytes7,bytes8,bytes9,bytes10,bytes11,bytes12,bytes13,bytes14,
    bytes15,bytes16,bytes17,bytes18,bytes19,bytes20,bytes21,bytes22,
    bytes23,bytes24,bytes25,bytes26,bytes27,bytes28,bytes29,bytes30,
    bytes31,bytes32,int,int8,int16,int24,int32,int40,int48,int56,int64,
    int72,int80,int88,int96,int104,int112,int120,int128,int136,int144,
    int152,int160,int168,int176,int184,int192,int200,int208,int216,int224,
    int232,int240,int248,int256,uint,uint8,uint16,uint24,uint32,uint40,
    uint48,uint56,uint64,uint72,uint80,uint88,uint96,uint104,uint112,uint120,
    uint128,uint136,uint144,uint152,uint160,uint168,uint176,uint184,uint192,
    uint200,uint208,uint216,uint224,uint232,uint240,uint248,uint256,wei,
    gwei,ether,seconds,minutes,hours,days,weeks,years
  },
  ndkeywordstyle = \color{ndSol}\bfseries,
  identifierstyle= \color{black},
  comment=[l]{//},
  morecomment=[s]{/*}{*/},
  commentstyle   = \color{commentFG}\ttfamily,
  stringstyle    = \color{stringFG}\ttfamily,
  morestring=[b]",
  morestring=[b]'
}

%% ---------------------------  TOML  ---------------------------
\lstdefinelanguage{toml}{
  sensitive=true,
  comment=[l]{\#},
  commentstyle     = \color{commentFG}\ttfamily,
  stringstyle      = \color{stringFG}\ttfamily,
  morestring=[b]",
  ndkeywords       = {true,false},
  ndkeywordstyle   = \color{ndRust}\bfseries,
  keywords         = {package,dependencies,channel,components,profile,version,edition,
                      resolver,members,features,default-features},
  keywordstyle     = \color{kwToml}\bfseries,
  identifierstyle  = \color{black}
}

%% ------------------------  Global style  -----------------------
\lstset{
  language      = rust,          % default
  basicstyle    = \ttfamily\small,
  columns       = fullflexible,
  keepspaces    = true,
  showstringspaces = false,
  tabsize       = 2,
  frame         = single,
  framerule     = 0.2pt,
  xleftmargin   = 1.5ex,
  xrightmargin  = 1.5ex,
  numbers       = left,
  numberstyle   = \scriptsize\color{commentFG},
  numbersep     = 6pt
}